\chapter{Protokolle}
\addtocontents{toc}{\protect\setcounter{tocdepth}{0}}
\section{Protokoll vom 24. Juli 2020}
\paragraph{Anwesende}
\begin{itemize}
	\item{\textbf{Prof. Thomas Stütz}}
	\item{\textbf{Halil Bahar}}
	\item{\textbf{Sonja Cao}}
\end{itemize}

\paragraph{Ort}
Sprachkonferenz über Discord

\paragraph{Inhalt}\mbox{}\\
Besprechung des Projektumfangs und der verwendeten Technologien

\section{Protokoll vom 03. August 2020}

\paragraph{Anwesende}
\begin{itemize}
	\item{\textbf{Prof. Thomas Stütz}}
	\item{\textbf{Halil Bahar}}
	\item{\textbf{Sonja Cao}}
\end{itemize}

\paragraph{Ort}
Sprachkonferenz über Discord

\paragraph{Inhalt}\mbox{}\\
Besprechung über die Erweiterung von Trix:
\begin{itemize}
	\item{Struktur des Projekts}
	\item{Verwendete Programmiersprachen sind JavaScript und TypeScript}
	\item{Bereitstellung der Barrierefreiheit}
\end{itemize}

\paragraph{Auftrag bis zum nächsten Meeting}\mbox{}\\
Erstellung einer Grafik für einen Überblick über die Schnittstellen und Beschreibung der notwendigen JavaScript Events.

\section{Protokoll vom 11. August 2020}

\paragraph{Anwesende}
\begin{itemize}
	\item{\textbf{Prof. Thomas Stütz}}
	\item{\textbf{Halil Bahar}}
	\item{\textbf{Sonja Cao}}
\end{itemize}

\paragraph{Ort}
Sprachkonferenz über Discord

\paragraph{Inhalt}\mbox{}\\
Besprechung der ersten schriftliche Ausarbeitung der Diplomarbeit:
\begin{itemize}
	\item{Bezeichnung der Kapitel}
	\item{Verbesserungsmöglichkeiten zur besseren Verständlichkeit der Arbeit}
\end{itemize}

\paragraph{Auftrag bis zum nächsten Meeting}\mbox{}\\
Verbesserung der aktuellen und Erstellung neuer Diagramme und Grafiken (Überblick über die Schnittstellen, DOM als Baumstruktur, Aussehen des Texteditors Trix) zum besseren Verständnis der Arbeit und Beginn der schriftlichen Ausarbeitung des Kapitel {\em{Einleitung}}.