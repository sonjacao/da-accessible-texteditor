\chapter{Protokolle}
\addtocontents{toc}{\protect\setcounter{tocdepth}{0}}
\section{Protokoll vom 24. Juli 2020}
\paragraph{Anwesende}
\begin{itemize}
	\item{\textbf{Prof. Thomas Stütz}}
	\item{\textbf{Halil Bahar}}
	\item{\textbf{Sonja Cao}}
\end{itemize}

\paragraph{Ort}\mbox{}\\
Sprachkonferenz über Discord

\paragraph{Inhalt}\mbox{}\\
Besprechung des Projektumfangs und der verwendeten Technologien

\section{Protokoll vom 03. August 2020}

\paragraph{Anwesende}
\begin{itemize}
	\item{\textbf{Prof. Thomas Stütz}}
	\item{\textbf{Halil Bahar}}
	\item{\textbf{Sonja Cao}}
\end{itemize}

\paragraph{Ort}\mbox{}\\
Sprachkonferenz über Discord

\paragraph{Inhalt}\mbox{}\\
Besprechung über die Erweiterung von Trix:
\begin{itemize}
	\item{Struktur des Projekts}
	\item{Verwendete Programmiersprachen sind JavaScript und TypeScript}
	\item{Bereitstellung der Barrierefreiheit}
\end{itemize}

\paragraph{Auftrag bis zum nächsten Meeting}\mbox{}\\
Erstellung einer Grafik für einen Überblick über die Schnittstellen und Beschreibung der notwendigen JavaScript Events.

\section{Protokoll vom 11. August 2020}

\paragraph{Anwesende}
\begin{itemize}
	\item{\textbf{Prof. Thomas Stütz}}
	\item{\textbf{Halil Bahar}}
	\item{\textbf{Sonja Cao}}
\end{itemize}

\paragraph{Ort}\mbox{}\\
Sprachkonferenz über Discord

\paragraph{Inhalt}\mbox{}\\
Besprechung der ersten schriftliche Ausarbeitung der Diplomarbeit:
\begin{itemize}
	\item{Bezeichnung der Kapitel}
	\item{Verbesserungsmöglichkeiten zur besseren Verständlichkeit der Arbeit}
\end{itemize}

\paragraph{Auftrag bis zum nächsten Meeting}\mbox{}\\
Verbesserung der aktuellen und Erstellung neuer Diagramme und Grafiken (Überblick über die Schnittstellen, DOM als Baumstruktur, Aussehen des Texteditors Trix) zum besseren Verständnis der Arbeit und Beginn der schriftlichen Ausarbeitung des Kapitel {\em{Einleitung}}.

\section{Protokoll vom 31. August 2020}

\paragraph{Anwesende}
\begin{itemize}
	\item{\textbf{Prof. Thomas Stütz}}
	\item{\textbf{Halil Bahar}}
	\item{\textbf{Sonja Cao}}
\end{itemize}

\paragraph{Ort}\mbox{}\\
Sprachkonferenz über Discord

\paragraph{Inhalt}\mbox{}\\
Besprechung der schriftlichen Ausarbeitung der Diplomarbeit:
\begin{itemize}
	\item{Korrigieren einzelner Rechtschreib- und Grammatikfehler}
	\item{Überlegungen über funktionale und nichtfunktionale Anforderungen}
\end{itemize}

\paragraph{Auftrag bis zum nächsten Meeting}\mbox{}\\
Beschreibung, warum der Texteditor in Fabasoft Produkten ersetzt werden muss, Verfassen kürzerer Sätze, Trennung zwischen Projektablauf und Produkt und Änderung der funktionalen und nichtfunktionalen Anforderungen.

\section{Protokoll vom 07. September 2020}

\paragraph{Anwesende}
\begin{itemize}
	\item{\textbf{Prof. Thomas Stütz}}
	\item{\textbf{Halil Bahar}}
\end{itemize}

\paragraph{Ort}\mbox{}\\
Sprachkonferenz über Discord

\paragraph{Inhalt}\mbox{}\\
Besprechung der schriftlichen Ausarbeitung der Diplomarbeit:
\begin{itemize}
	\item{Systemarchitektur mit und ohne der Erweiterung}
\end{itemize}

\paragraph{Auftrag bis zum nächsten Meeting}\mbox{}\\
Skizzieren der Systemarchitektur mit und ohne Erweiterung, Beschreibung der Zusammenhänge und Unterschiede und Verbessern der Beschreibung beim Kapitel ``Funktionale Anforderungen``.

\section{Protokoll vom 21. September 2020}

\paragraph{Anwesende}
\begin{itemize}
	\item{\textbf{Prof. Thomas Stütz}}
	\item{\textbf{Halil Bahar}}
	\item{\textbf{Sonja Cao}}
\end{itemize}

\paragraph{Ort}\mbox{}\\
Sprachkonferenz über Discord

\paragraph{Inhalt}\mbox{}\\
Besprechung der schriftlichen Ausarbeitung der Diplomarbeit:
\begin{itemize}
	\item{Kleine Grammatik- und Rechtschreibfehler}
	\item{Struktur der Kapitel}
	\item{Platzierung und Beschreibung der Systemarchitektur und des Texteditors}
\end{itemize}

\paragraph{Auftrag bis zum nächsten Meeting}\mbox{}\\
Ausbessern der Fehler, Zusätzliches Kapitel für erfüllte WCAG 2.1 Kriterien, Kapitel für Web Components

\section{Protokoll vom 09. November 2020}

\paragraph{Anwesende}
\begin{itemize}
	\item{\textbf{Prof. Thomas Stütz}}
	\item{\textbf{Halil Bahar}}
	\item{\textbf{Sonja Cao}}
\end{itemize}

\paragraph{Ort}\mbox{}\\
Sprachkonferenz über Discord

\paragraph{Inhalt}\mbox{}\\
Besprechung der schriftlichen Ausarbeitung der Diplomarbeit:
\begin{itemize}
	\item{Beschreibung der Begriffe Design Patterns und Factory Method Pattern}
	\item{Überarbeitung des Kapitels 3.5 JavaScript Events \& EventListener}
\end{itemize}

\paragraph{Auftrag bis zum nächsten Meeting}\mbox{}\\
Ergänzung des Factory Method Patterns mit einem UML-Klassendiagramm

\section{Protokoll vom 30. November 2020}

\paragraph{Anwesende}
\begin{itemize}
	\item{\textbf{Prof. Thomas Stütz}}
	\item{\textbf{Halil Bahar}}
	\item{\textbf{Sonja Cao}}
\end{itemize}

\paragraph{Ort}\mbox{}\\
Sprachkonferenz über Discord

\paragraph{Inhalt}\mbox{}\\
Besprechung der schriftlichen Ausarbeitung der Diplomarbeit:
\begin{itemize}
	\item UML-Abbildung des Factory Method Patterns: Methode \texttt{createProduct()} muss \texttt{static} sein
	\item Kapitel Git: Merging beschrieben
	\item Definieren der Arbeitsaufteilung
	\item Eventuell Beispiele für WCAG: Webseiten, die die Richtlinien erfüllen und Webseiten, die diese nicht 
		erfüllen	
\end{itemize}

\paragraph{Auftrag bis zum nächsten Meeting}\mbox{}\\
Fertigstellen aller Kapitel so weit wie möglich bis auf Kapitel Technologien