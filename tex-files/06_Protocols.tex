\chapter{Protokolle}
\addtocontents{toc}{\protect\setcounter{tocdepth}{0}}
% ----------- Protokoll Fabasoft ----------- %
\section{Meetings in der Firma Fabasoft}
\subsection{Protokoll vom 08. Juli 2020}
\paragraph{Anwesende}
\begin{itemize}
	\item{\textbf{Thomas Bühringer}}
	\item{\textbf{Halil Bahar}}
	\item{\textbf{Sonja Cao}}
\end{itemize}

\paragraph{Ort}\mbox{}\\
Meetingraum

\paragraph{Inhalt}
\begin{itemize}
	\item Besprechung der Arbeit (Ist-Zustand, Problemstellung, Aufgabenstellung, Zielsetzung)
	\item Terminvereinbarung für einen Crashkurs in ''Web Accessibility''
\end{itemize}

\subsection{Protokoll vom 09. Juli 2020}
\paragraph{Anwesende}
\begin{itemize}
	\item{\textbf{Mario Batusic}}
	\item{\textbf{Halil Bahar}}
	\item{\textbf{Sonja Cao}}
\end{itemize}

\paragraph{Ort}\mbox{}\\
Büro

\paragraph{Inhalt}
\begin{itemize}
	\item Crashkurs in ''Web Accessibility''
	\item Besprechung der Anforderungen an einen barrierefreien Texteditor
\end{itemize}

\subsection{Protokoll vom 15. Juli 2020}
\paragraph{Anwesende}
\begin{itemize}
	\item{\textbf{Mario Batusic}}
	\item{\textbf{Halil Bahar}}
	\item{\textbf{Sonja Cao}}
\end{itemize}

\paragraph{Ort}\mbox{}\\
Büro

\paragraph{Inhalt}\mbox{}\\
Mario testet die erste Implementierung zur Nutzung der Toolbar über die Tastatur und gibt sein Feedback dazu.

\subsection{Protokoll vom 16. Juli 2020}
\paragraph{Anwesende}
\begin{itemize}
	\item{\textbf{Mario Batusic}}
	\item{\textbf{Halil Bahar}}
	\item{\textbf{Sonja Cao}}
\end{itemize}

\paragraph{Ort}\mbox{}\\
Büro

\paragraph{Inhalt}
\begin{itemize}
	\item Funktionalität der Pfeiltasten vertauscht (z. B. Die Pfeiltaste nach oben sollte dieselbe Funktion haben wie die Pfeiltaste nach rechts.)
	\item Besprechung der fehlenden HTML-Attribute für die Barrierefreiheit
\end{itemize}

\subsection{Protokoll vom 17. Juli 2020}
\paragraph{Anwesende}
\begin{itemize}
	\item{\textbf{Mario Batusic}}
	\item{\textbf{Halil Bahar}}
	\item{\textbf{Sonja Cao}}
\end{itemize}

\paragraph{Ort}\mbox{}\\
Büro

\paragraph{Inhalt}
\begin{itemize}
	\item Verwendung der \texttt{Tab}-Taste sollte nicht ermöglicht werden
	\item HTML-Attribut \texttt{alt} für \texttt{<img>}-Tags einpflegen	
\end{itemize}

\subsection{Protokoll vom 17. Juli 2020}
\paragraph{Anwesende}
\begin{itemize}
	\item{\textbf{Thomas Bühringer}}
	\item{\textbf{Halil Bahar}}
	\item{\textbf{Sonja Cao}}
\end{itemize}

\paragraph{Ort}\mbox{}\\
Meetingraum

\paragraph{Inhalt}\mbox{}\\
Besprechung der geplanten weiteren Vorgehensweise:
\begin{enumerate}
	\item vorhandene Tools des Texteditors barrierefrei machen
	\item Überlegung, welche Buttons implementiert werden sollen und wie
	\item Pull Request machen, nachdem alle Funktionalitäten implementiert worden sind
\end{enumerate}

\subsection{Protokoll vom 21. Juli 2020}
\paragraph{Anwesende}
\begin{itemize}
	\item{\textbf{Mario Batusic}}
	\item{\textbf{Halil Bahar}}
	\item{\textbf{Sonja Cao}}
\end{itemize}

\paragraph{Ort}\mbox{}\\
Büro

\paragraph{Inhalt}\mbox{}\\
Tools des CKEditors sollten in Trix ebenfalls ermöglich werden

\subsection{Protokoll vom 23. Juli 2020}
\paragraph{Anwesende}
\begin{itemize}
	\item{\textbf{Thomas Bühringer}}
	\item{\textbf{Mateusz Szostak}}
	\item{\textbf{Halil Bahar}}
	\item{\textbf{Sonja Cao}}
\end{itemize}

\paragraph{Ort}\mbox{}\\
Büro

\paragraph{Inhalt}
\begin{itemize}
	\item Planung zur Erstellung der eigenen Erweiterung, da keine Rückmeldung von den Entwicklern von Trix gekommen ist
	\item Besprechung der Schwierigkeiten des Event Handling von Trix
	\item nächstes Ziel ist die Erweiterung des DOMs mit weiteren HTML-Tags
	\item Besprechung welche Tools in die Toolbar eingepflegt werden sollen
	\item Dialogfenster sollten erstellt werden können
\end{itemize}

\subsection{Protokoll vom 29. Juli 2020}
\paragraph{Anwesende}
\begin{itemize}
	\item{\textbf{Mario Batusic}}
	\item{\textbf{Halil Bahar}}
	\item{\textbf{Sonja Cao}}
\end{itemize}

\paragraph{Ort}\mbox{}\\
Büro

\paragraph{Inhalt}\mbox{}\\
Besprechung welche Funktionalitäten nun möglich sind und welche fehlen

\subsection{Protokoll vom 29. Juli 2020}
\paragraph{Anwesende}
\begin{itemize}
	\item{\textbf{Thomas Bühringer}}
	\item{\textbf{Mateusz Szostak}}
	\item{\textbf{Halil Bahar}}
	\item{\textbf{Sonja Cao}}
\end{itemize}

\paragraph{Ort}\mbox{}\\
Büro

\paragraph{Inhalt}
\begin{itemize}
	\item Besprechung der bisherigen Implementierungen
	\item Besprechung der weiteren Vorgehensweise:
		\begin{enumerate}
			\item Darstellung von Tabellen ermöglichen
			\item Erstellung von Dialogfenstern mit Buttons, um zurück zur Toolbar oder zum Editor zu gelangen
			\item Beispiel einer Toolbar bereitstellen
		\end{enumerate}
\end{itemize}

\subsection{Meetings im Zeitraum von 03. August bis 28. August 2020}
\paragraph{Anwesende}
\begin{itemize}
	\item{\textbf{Thomas Bühringer}}
	\item{\textbf{Mateusz Szostak}}
	\item{\textbf{Halil Bahar}}
	\item{\textbf{Sonja Cao}}
\end{itemize}

\paragraph{Ort}\mbox{}\\
Meetingraum

\paragraph{Inhalt}
\begin{itemize}
	\item Besprechung der Anforderungen an die Funktionalitäten zur Implementierung des Texteditors in der Fabasoft Cloud
	\item Bugfixing
	\item Korrekturen zur richtigen Darstellung der Textformatierungen
	\item Darstellung aller HTML-Tags ermöglichen
\end{itemize}

% ----------- Protokoll Stütz ----------- %
\section{Meetings mit Prof. Thomas Stütz}
\subsection{Protokoll vom 24. Juli 2020}
\paragraph{Anwesende}
\begin{itemize}
	\item{\textbf{Prof. Thomas Stütz}}
	\item{\textbf{Halil Bahar}}
	\item{\textbf{Sonja Cao}}
\end{itemize}

\paragraph{Ort}\mbox{}\\
Sprachkonferenz über Discord

\paragraph{Inhalt}\mbox{}\\
Besprechung des Projektumfangs und der verwendeten Technologien

\subsection{Protokoll vom 03. August 2020}

\paragraph{Anwesende}
\begin{itemize}
	\item{\textbf{Prof. Thomas Stütz}}
	\item{\textbf{Halil Bahar}}
	\item{\textbf{Sonja Cao}}
\end{itemize}

\paragraph{Ort}\mbox{}\\
Sprachkonferenz über Discord

\paragraph{Inhalt}\mbox{}\\
Besprechung über die Erweiterung von Trix:
\begin{itemize}
	\item{Struktur des Projekts}
	\item{Verwendete Programmiersprachen sind JavaScript und TypeScript}
	\item{Bereitstellung der Barrierefreiheit}
\end{itemize}

\paragraph{Auftrag bis zum nächsten Meeting}\mbox{}\\
Erstellung einer Grafik für einen Überblick über die Schnittstellen und Beschreibung der notwendigen JavaScript Events.

\subsection{Protokoll vom 11. August 2020}

\paragraph{Anwesende}
\begin{itemize}
	\item{\textbf{Prof. Thomas Stütz}}
	\item{\textbf{Halil Bahar}}
	\item{\textbf{Sonja Cao}}
\end{itemize}

\paragraph{Ort}\mbox{}\\
Sprachkonferenz über Discord

\paragraph{Inhalt}\mbox{}\\
Besprechung der ersten schriftliche Ausarbeitung der Diplomarbeit:
\begin{itemize}
	\item{Bezeichnung der Kapitel}
	\item{Verbesserungsmöglichkeiten zur besseren Verständlichkeit der Arbeit}
\end{itemize}

\paragraph{Auftrag bis zum nächsten Meeting}\mbox{}\\
Verbesserung der aktuellen und Erstellung neuer Diagramme und Grafiken (Überblick über die Schnittstellen, DOM als Baumstruktur, Aussehen des Texteditors Trix) zum besseren Verständnis der Arbeit und Beginn der schriftlichen Ausarbeitung des Kapitel {\em{Einleitung}}.

\subsection{Protokoll vom 31. August 2020}

\paragraph{Anwesende}
\begin{itemize}
	\item{\textbf{Prof. Thomas Stütz}}
	\item{\textbf{Halil Bahar}}
	\item{\textbf{Sonja Cao}}
\end{itemize}

\paragraph{Ort}\mbox{}\\
Sprachkonferenz über Discord

\paragraph{Inhalt}\mbox{}\\
Besprechung der schriftlichen Ausarbeitung der Diplomarbeit:
\begin{itemize}
	\item{Korrigieren einzelner Rechtschreib- und Grammatikfehler}
	\item{Überlegungen über funktionale und nichtfunktionale Anforderungen}
\end{itemize}

\paragraph{Auftrag bis zum nächsten Meeting}\mbox{}\\
Beschreibung, warum der Texteditor in Fabasoft Produkten ersetzt werden muss, Verfassen kürzerer Sätze, Trennung zwischen Projektablauf und Produkt und Änderung der funktionalen und nichtfunktionalen Anforderungen.

\subsection{Protokoll vom 07. September 2020}

\paragraph{Anwesende}
\begin{itemize}
	\item{\textbf{Prof. Thomas Stütz}}
	\item{\textbf{Halil Bahar}}
\end{itemize}

\paragraph{Ort}\mbox{}\\
Sprachkonferenz über Discord

\paragraph{Inhalt}\mbox{}\\
Besprechung der schriftlichen Ausarbeitung der Diplomarbeit:
\begin{itemize}
	\item{Systemarchitektur mit und ohne der Erweiterung}
\end{itemize}

\paragraph{Auftrag bis zum nächsten Meeting}\mbox{}\\
Skizzieren der Systemarchitektur mit und ohne Erweiterung, Beschreibung der Zusammenhänge und Unterschiede und Verbessern der Beschreibung beim Kapitel ``Funktionale Anforderungen``.

\subsection{Protokoll vom 21. September 2020}

\paragraph{Anwesende}
\begin{itemize}
	\item{\textbf{Prof. Thomas Stütz}}
	\item{\textbf{Halil Bahar}}
	\item{\textbf{Sonja Cao}}
\end{itemize}

\paragraph{Ort}\mbox{}\\
Sprachkonferenz über Discord

\paragraph{Inhalt}\mbox{}\\
Besprechung der schriftlichen Ausarbeitung der Diplomarbeit:
\begin{itemize}
	\item{Kleine Grammatik- und Rechtschreibfehler}
	\item{Struktur der Kapitel}
	\item{Platzierung und Beschreibung der Systemarchitektur und des Texteditors}
\end{itemize}

\paragraph{Auftrag bis zum nächsten Meeting}\mbox{}\\
Ausbessern der Fehler, Zusätzliches Kapitel für erfüllte WCAG 2.1 Kriterien, Kapitel für Web Components

\subsection{Protokoll vom 09. November 2020}

\paragraph{Anwesende}
\begin{itemize}
	\item{\textbf{Prof. Thomas Stütz}}
	\item{\textbf{Halil Bahar}}
	\item{\textbf{Sonja Cao}}
\end{itemize}

\paragraph{Ort}\mbox{}\\
Sprachkonferenz über Discord

\paragraph{Inhalt}\mbox{}\\
Besprechung der schriftlichen Ausarbeitung der Diplomarbeit:
\begin{itemize}
	\item{Beschreibung der Begriffe Design Patterns und Factory Method Pattern}
	\item{Überarbeitung des Kapitels 3.5 JavaScript Events \& EventListener}
\end{itemize}

\paragraph{Auftrag bis zum nächsten Meeting}\mbox{}\\
Ergänzung des Factory Method Patterns mit einem UML-Klassendiagramm

\subsection{Protokoll vom 30. November 2020}

\paragraph{Anwesende}
\begin{itemize}
	\item{\textbf{Prof. Thomas Stütz}}
	\item{\textbf{Halil Bahar}}
	\item{\textbf{Sonja Cao}}
\end{itemize}

\paragraph{Ort}\mbox{}\\
Sprachkonferenz über Discord

\paragraph{Inhalt}\mbox{}\\
Besprechung der schriftlichen Ausarbeitung der Diplomarbeit:
\begin{itemize}
	\item UML-Abbildung des Factory Method Patterns: Methode \texttt{createProduct()} muss \texttt{static} sein
	\item Kapitel Git: Merging beschrieben
	\item Definieren der Arbeitsaufteilung
	\item Eventuell Beispiele für WCAG: Webseiten, die die Richtlinien erfüllen und Webseiten, die diese nicht 
		erfüllen	
\end{itemize}

\paragraph{Auftrag bis zum nächsten Meeting}\mbox{}\\
Fertigstellen aller Kapitel so weit wie möglich bis auf Kapitel Technologien

\subsection{Protokoll vom 14. Februar 2021}

\paragraph{Anwesende}
\begin{itemize}
	\item{\textbf{Prof. Thomas Stütz}}
	\item{\textbf{Halil Bahar}}
	\item{\textbf{Sonja Cao}}
\end{itemize}

\paragraph{Ort}\mbox{}\\
Sprachkonferenz über Discord

\paragraph{Inhalt}\mbox{}\\
Besprechung der schriftlichen Ausarbeitung der Diplomarbeit:
\begin{itemize}
	\item Fragen bezüglich Kapitel ``Verwendete Technologien``geklärt
\end{itemize}

\paragraph{Auftrag bis zum nächsten Meeting}\mbox{}\\
Beschreiben von JavaScript und TypeScript

\subsection{Protokoll vom 9. April 2021}

\paragraph{Anwesende}
\begin{itemize}
	\item{\textbf{Prof. Thomas Stütz}}
	\item{\textbf{Halil Bahar}}
	\item{\textbf{Sonja Cao}}
\end{itemize}

\paragraph{Ort}\mbox{}\\
Sprachkonferenz über Discord

\paragraph{Inhalt}\mbox{}\\
Besprechung der schriftlichen Ausarbeitung der Diplomarbeit:
\begin{itemize}
	\item Besprechen des Kapitels ``Verwendete Technologien``
\end{itemize}

\paragraph{Auftrag bis zum nächsten Meeting}\mbox{}\\
Beschreiben von npm, TypeScript und Webpack, Kapitel ``Ausgewählte Aspekte``ergänzen, Zitieren von Webseiten und Büchern