\section*{Declaration of Academic Honesty} % * ... section numbering
Hereby, I declare that I have composed the presented paper independently on my own and without any other resources than the ones indicated. All thoughts taken directly or indirectly from external sources are properly denoted as such.

This paper has neither been previously submitted to another authority nor has it been published yet. \\[1em]
Leonding, \duedateen \\[5em]
\ifthenelse{\isundefined{\firstauthor}}{}{\firstauthor}
\ifthenelse{\isundefined{\secondauthor}}{}{\kern-1ex, \secondauthor}
\ifthenelse{\isundefined{\thirdauthor}}{}{\kern-1ex, \thirdauthor}
\ifthenelse{\isundefined{\fourthauthor}}{}{\kern-1ex, \fourthauthor} \\[5em]

\begin{otherlanguage}{german}
\section*{Eidesstattliche Erklärung}
Hiermit erkläre ich an Eides statt, dass ich die vorgelegte Diplomarbeit selbstständig und ohne Benutzung anderer als der angegebenen Hilfsmittel angefertigt habe. Gedanken, die aus fremden Quellen direkt oder indirekt übernommen wurden, sind als solche gekennzeichnet.

Die Arbeit wurde bisher in gleicher oder ähnlicher Weise keiner anderen Prüfungsbehörde vorgelegt und auch noch nicht veröffentlicht. \\[1em]
Leonding, am \duedatede \\[5em]
\ifthenelse{\isundefined{\firstauthor}}{}{\firstauthor}
\ifthenelse{\isundefined{\secondauthor}}{}{\kern-1ex, \secondauthor}
\ifthenelse{\isundefined{\thirdauthor}}{}{\kern-1ex, \thirdauthor}
\ifthenelse{\isundefined{\fourthauthor}}{}{\kern-1ex, \fourthauthor} \\[5em]
\end{otherlanguage}

\section*{Gender Declaration} % * ... section numbering
In this diploma thesis the language form of the generic masculine is used for reasons of better readability. All personal designations are therefore to be understood as gender-neutral.

\begin{otherlanguage}{german}
\section*{Gender Erklärung}
Aus Gründen der besseren Lesbarkeit wird in dieser Diplomarbeit die Sprachform des generischen Maskulinums verwendet. Alle personenbezogenen Bezeichnungen sind somit geschlechtsneutral zu verstehen.
\end{otherlanguage}

\newcounter{abstractpage}
\setcounter{abstractpage}{\value{page}}

\begin{otherlanguage}{english}
\begin{abstract} % environment 'abstract'
In the course of the diploma thesis of Halil Bahar and Sonja Cao an extension for the open source What You See Is What You Get (WYSIWYG) text editor Trix from the company Basecamp is being developed for the client Fabasoft Software GmbH. The previous text editor of the Fabasoft Cloud no longer meets the desired requirements and Trix is still unsuitable in terms of accessibility for people with disabilities. The extension is therefore intended to be used in products of Fabasoft and to meet the criteria of WCAG 2.1, making it accessible with screen readers and the keyboard to enable all users to write and format texts without barriers.

\end{abstract}
\end{otherlanguage}

\begin{otherlanguage}{german}
\begin{abstract}
Im Zuge der Diplomarbeit von Halil Bahar und Sonja Cao wird der Open Source What You See Is What You Get (WYSIWYG) Texteditor namens Trix vom Unternehmen Basecamp für den Auftraggeber Fabasoft Software GmbH erweitert. Der bisherige Texteditor der Fabasoft Cloud entspricht nicht mehr den gewünschten Anforderungen und Trix ist hinsichtlich der Accessibility für Menschen mit Behinderungen noch ungeeignet. Die Erweiterung soll daher einen Einsatz in Fabasoft Produkten ermöglichen sowie die Kriterien der WCAG 2.1 erfüllen und so mit Screenreadern sowie Tastatur zugänglich sein, um allen Benutzern die Möglichkeit zu geben, Texte barrierefrei zu verfassen und zu formatieren.

\end{abstract}
\end{otherlanguage}

\setcounter{page}{\value{abstractpage}}
\stepcounter{page}

\section*{Danksagung}
Wir möchten uns bei allen Personen bedanken, die uns bei unserer Arbeit unterstützt haben, um ein erfolgreiches Ergebnis zu liefern:

%\begin{itemize}
%	\item 
%\end{itemize}

