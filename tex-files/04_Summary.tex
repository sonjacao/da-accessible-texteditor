\chapter{Resümee}
\section{Halil Bahar}
Mit voller Überzeugung und Motivation habe ich mein Praktikum in der Firma Fabasoft begonnen. Seit dem ersten Tag an, durfte ich viele tolle und hilfsbereite Kolleginnen und Kollegen kennenlernen, die uns bis zum Ende des Praktikums begleitet und unterstützt haben.\\
Am Anfangs des Praktikums lag der Fokus nicht am Programmieren, sondern auf der Vorbereitung für die nächsten Wochen. Wir durften an einem Workshop über Web Accessibility teilnehmen, was mir große Freude bereitet hat.

\mbox{}\\
Anschließend hat das Programmieren begonnen. In den Anfangszeiten hatten wir öfters Probleme, was die Architektur anbelangt. Wir haben lange und detailliert darüber diskutiert, welche weiteren Schritte notwendig sind. Durch diese Probleme konnte ich viele Ideen und Lösungswegen mit meinem Team besprechen und austauschen. Dieser Lernprozess hat mir besonders gut gefallen.\\
Besonders stolz war ich darauf, dass wir unsere Software rechtzeitig fertiggestellt und die letzten Wochen damit verbracht haben, unsere Software in die Fabasoft Cloud zu integrieren.


\section{Sonja Cao}
Zuversichtlich, dass wir schnell fertig werden, haben wir uns an die Arbeitsplätze gesetzt. Allerdings hat das Einarbeiten und das Aufsetzen der passenden Entwicklungsumgebungen sowie die Sammlung wichtiger Vorkenntnisse über Web Accessibility etwa eine Woche in Anspruch genommen. Nachdem dieser Schritt geschafft wurde, hat auch die tatsächliche Arbeit begonnen. \\
Mit der Unterstützung des Teams der Firma Fabasoft haben wir den besten Lösungsweg ausfindig machen können, um unser Ziel zu erreichen. Mithilfe der zahlreichen Ratschläge und Feedbacks ist es möglich gewesen Probleme schnell zu beheben, falls vorhanden, und die gewünschten Anforderungen einzupflegen. 

\mbox{}\\
Insbesondere das Ziel, einen WYSIWYG Texteditor barrierefrei zu machen, macht die Diplomarbeit zu etwas Besonderem. Dadurch können viele Menschen, unabhängig von ihrer physischen oder psychischen Einschränkungen, den Texteditor problemlos nutzen. Der Gedanke daran, dieses Ziel erreichen zu wollen, hat mich dazu motiviert, so exakt wie möglich zu arbeiten und immer mein bestes zu geben.\\
Ebenso hat mir das Auseinandersetzen mit diesem umfangreichen Thema noch viel deutlicher gemacht, dass sich viele Menschen mit Behinderungen noch benachteiligt fühlen. Es gibt noch immer zahlreiche Personen, die nicht ihr volles Potenzial entfalten können. Aus diesem Grund werde ich von nun an noch deutlicher darauf achten, dass eine barrierefreie Nutzung jeglicher Webseiten, Applikationen und sonstige Dinge wichtig ist.