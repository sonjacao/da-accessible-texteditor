\chapter{Resümee}
\section{Halil Bahar}

\section{Sonja Cao}
Mit großer Zuversicht, dass wir schnell fertig werden, haben wir uns an die Arbeitsplätze gesetzt. Allerdings hat das Einarbeiten und das Aufsetzen der passenden Entwicklungsumgebungen sowie die Sammlung wichtiger Vorkenntnisse über Web Accessibility etwa eine Woche in Anspruch genommen. Nachdem dieser Schritt geschafft wurde, hat auch die tatsächliche Arbeit begonnen. \\
Der anfängliche Plan, den Quellcode der Entwickler des WYSIWYG Texteditors Trix mit einigen Erweiterungen zur Erreichung der Barrierefreiheit zu verändern, hat nicht so funktioniert wie vorgesehen. Dennoch haben wir mit der Unterstützung des Teams der Firma Fabasoft einen anderen Lösungsweg gefunden, um das Ziel zu erreichen. Mit zahlreichen Ratschlägen und Feedbacks ist es möglich gewesen Probleme schnell zu beheben und die gewünschten Anforderungen einzupflegen. 

\mbox{}\\
Insbesondere das Ziel, einen WYSIWYG Texteditor barrierefrei zu machen, macht die Diplomarbeit zu etwas Besonderem. Dadurch können viele Menschen, unabhängig von ihrer physischen oder psychischen Einschränkungen, den Texteditor problemlos nutzen. Der Gedanke daran, dieses Ziel erreichen zu wollen, hat mich dazu motiviert, so exakt wie möglich zu arbeiten und immer mein bestes zu geben. 